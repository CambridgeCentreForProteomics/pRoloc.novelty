\section*{Foreword}

%% \Rpackage{MSnbase} is in an early development %%(package version \Sexpr{packageVersion("MSnbase")}, 
%% (see section \ref{sec:sessionInfo} for details about packages and version used in this vignette). 
%% Although main data structures have been thought out and are meant to be compatible with other existing 
%% mature infrastructure in the Bioconductor project, changes may occur in the future. 
%% Current functionality will evolve and new one will be added. 
%% Although at an early stage, this package is released in the hope that it may foster  
%% new developments in proteomics data analysis within \R by providing a common infrastructure. 
%% Several package developers working with mass spectrometry and proteomics data met at the 
%% Bioconductor Developer Meeting Europe\footnote{\url{http://bioconductor.org/help/course-materials/2010/HeidelbergNovember2010/}} 
%% held in Heidelberg in November 2010, and agreed to combine efforts. 
%% This library is one attempt to do so.

\Rpackage{MSnbase} and \Rpackage{pRoloc} are under active developed; 
current functionality is evolving and new features will be added. 
This software is free and open-source software. 
If you use it, please support the project by citing it in publications:

\begin{quote}
  Laurent Gatto and Kathryn S. Lilley. \emph{MSnbase - an R/Bioconductor 
    package for isobaric tagged mass spectrometry data visualization,
    processing and quantitation.} Bioinformatics 28, 288-289 (2011).
\end{quote}

If you are using the \Rfunction{phenoDisco} function, please also cite

\begin{quote}
  Lisa M Breckels, Laurent Gatto, Andy Christoforou, Arnoud J Groen, 
  Kathryn S Lilley and Matthew W Trotter. 
  \emph{The effect of organelle discovery upon sub-cellular protein localisation.} 
  J Proteomics, S1874-3919(13)00094-8 (2013)
\end{quote}

You are welcome to contact us for questions, bugs, typos or suggestions about the packages.
If you wish to reach a broader audience for general questions about proteomics analysis using 
\texttt{R}, you may want to use the Bioconductor mailing 
list\footnote{\url{https://stat.ethz.ch/mailman/listinfo/bioconductor}}.
